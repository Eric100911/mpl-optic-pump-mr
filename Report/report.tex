\documentclass[draft]{Report/thuemp}
% \documentclass{Report/thuemp}
\begin{document}

% User-defined symbols
\newcommand{\rb}{\mathrm{rb}}
\newcommand{\rbeightyfive}{{}^{85}\mathrm{Rb}}
\newcommand{\rbeightyseven}{{}^{87}\mathrm{Rb}}
\newcommand{\fineFiveS}{$5^2S_{1/2}$}
\newcommand{\fineFiveP}{$5^2P_{1/2}$}

% 标题,作者
\emptitle{光泵磁共振}
\empauthor{你的姓名}{你的学号} 

% 奇数页页眉
\fancyhead[CO]{{\footnotesize 你的姓名: 光泵磁共振}}

%%%%%%%%%%%%%%%%%%%%%%%%%%%%%%%%%%%%%%%%%%%%%%%%%%%%%%%%%%%%%%%%
% 关键词 摘要 首页脚注
%%%%%%%%关键词
\Keyword{光抽运,磁共振,塞曼分裂,朗德因子,地磁场}
\twocolumn[
\begin{@twocolumnfalse}
\maketitle

%%%%%%%%摘要
\begin{empAbstract}
本实验利用光抽运技术,通过圆偏振的 $D_1$ 光与铷(Rb)原子相互作用,实现了原子在基态塞曼子能级上的粒子数偏极化,并观测到了光抽运信号。实验首先通过调节水平和垂直方向的补偿磁场,抵消了地磁场的影响,测量并分析了光抽运信号强度随磁场变化的规律,从而测定了地磁场的垂直分量及水平分量。在此基础上,施加频率为 $600\ \mathrm{kHz}$ 的射频场,在满足磁共振条件时诱导原子在塞曼子能级间跃迁,破坏偏极化状态,从而观测到磁共振信号。通过扫场法精确测量了 $\rbeightyfive$ 和 $\rbeightyseven$ 同位素发生磁共振时的共振磁场,计算得到其朗德 $g_F$ 因子分别为 $0.334$ 和 $0.501$(此处请填入你的实际实验值),与理论值符合较好。实验结果直观展示了光与原子相互作用的量子特性以及利用光探测技术进行微弱信号检测的高灵敏度。
\end{empAbstract}

%%%%%%%%英文标题、作者、摘要、关键词
\emptitleEn{Optical Pumping and Magnetic Resonance}
\empauthorEn{Your Name}{Your Student ID}
\KeywordEn{Optical Pumping, Magnetic Resonance, Zeeman Splitting, Landé g-factor, Geomagnetic Field}

\begin{empAbstractEn}
In this experiment, optical pumping technology was employed to achieve population polarization of Rubidium (Rb) atoms in ground state Zeeman sublevels through the interaction with circularly polarized $D_1$ light. 
The optical pumping signals were successfully observed. 
By adjusting the compensation magnetic fields in both horizontal and vertical directions to cancel the geomagnetic field, the relationship between the optical pumping signal intensity and the magnetic field was measured and analyzed, allowing for the determination of the vertical and horizontal components of the geomagnetic field. 
Subsequently, a radio-frequency (RF) field of $600\ \mathrm{kHz}$ was applied. 
When the magnetic resonance condition was met, transitions between Zeeman sublevels were induced, destroying the polarization state and producing magnetic resonance signals. 
Using the field-sweeping method, the resonance magnetic fields for $\rbeightyfive$ and $\rbeightyseven$ isotopes were precisely measured. 
The calculated Landé $g_F$ factors were $0.334$ and $0.501$ respectively (Fill in your data), which agreed well with theoretical values. 
The results vividly demonstrate the quantum nature of light-atom interactions and the high sensitivity of optical detection techniques for weak signals.
\end{empAbstractEn}

%%%%%%%%首页角注
\empfirstfoot{2025-xx-xx}{2025-xx-xx}{你的学号}{你的邮箱}
\end{@twocolumnfalse}
]

%%%%%%%%%%%%%%%%%%%%%%%%%%%%%%%%%%%%%%%%%%%%%%%%%%%%%%%%%%%%%%%%
%  正文由此开始
\wuhao 
 
\section{引言}
在物理学研究中,为了克服传统光谱学方法受限于仪器分辨率和多普勒展宽的问题,波谱学方法应运而生。然而,对于气态原子而言,由于样品浓度低,热平衡下能级粒子布居数差极小,导致共振信号微弱。20 世纪 50 年代,Kastler 等人提出了光抽运(Optical Pumping)概念,利用圆偏振光打破原子的玻尔兹曼热平衡分布,造成极大的布居数差,并结合“光探测”技术,极大地提高了探测灵敏度。

本实验旨在通过光泵磁共振实验系统,研究铷原子的能级结构及塞曼分裂现象,掌握光抽运和光探测的基本原理,并利用磁共振技术测量铷同位素的朗德 $g_F$ 因子及实验室环境的地磁场。

\section{实验仪器}

实验主体装置主要由铷光谱灯、光学系统、吸收池及磁场线圈系统组成,如图 \ref{fig:setup} 所示。

\begin{figure}[htbp]
    \centering
    % 请替换为实际的仪器示意图,或使用 TikZ 绘制简图
    % \includegraphics[width=0.45\textwidth]{Data/setup_schematic.png}
    \fbox{\parbox{0.4\textwidth}{\centering \vspace{2cm} [此处插入实验装置原理图] \vspace{2cm}}}
    \captionnamefont{\wuhao\bf\heiti}
    \captiontitlefont{\wuhao\bf\heiti}
    \caption{光泵磁共振实验光路及装置示意图}
    \label{fig:setup}
\end{figure}

光源采用由高频振荡器激励的铷原子光谱灯,其发射的光谱中包含 $D_1$ ($794.8\ \mathrm{nm}$) 和 $D_2$ ($780.0\ \mathrm{nm}$) 线。为了实现有效的光抽运,光路中设置了干涉滤光片滤除 $D_2$ 线,仅保留 $D_1$ 线。随后,光束经过准直透镜、起偏器和 $\lambda/4$ 波片,转变为$\sigma^+$圆偏振光。

吸收池置于两对亥姆霍兹线圈的中心,内部充有铷蒸汽(含 $\rbeightyfive$ 和 $\rbeightyseven$)及缓冲气体氮气。线圈系统包括垂直线圈、水平线圈:垂直线圈用于抵消地磁场垂直分量,水平线圈提供塞曼分裂的主磁场 $B_{\parallel}$,以及方波或三角波调制磁场 $B_s$ 作为扫场。此外,还有一对射频线圈用于提供频率为 $\nu$ 的射频场,以激发磁共振跃迁。透射光强由光电池及放大器系统检测,并输送至示波器进行观察。

实验中使用的 C 组仪器参数如下:水平线圈匝数 $N=250$,有效半径 $r=0.2410\ \mathrm{m}$;垂直线圈匝数 $N=100$,有效半径 $r=0.1530\ \mathrm{m}$。

\section{实验内容与方法}

\subsection{光抽运信号的观察与地磁场测量}

源自铷灯的$D_1$线光,若仅从能级间隔看,能够诱导$\rbeightyfive, \rbeightyseven$从$\fineFiveS$的各个超精细结构能级到$\fineFiveP$的各个超精细结构能级的跃迁,受到激发的$\rb$原子随后发生自发辐射。通过反复激发与辐射,这些原子原则上可去到所有的超精细结构子态。

在本实验中进行光抽运时,铷样品周围有一定的磁场存在,而入射光被制备为具有$\sigma^+$圆偏振;此时经电偶极跃迁发生的激发过程均满足$\Delta m_F=+1$,而相应发生的自发辐射过程总满足$\Delta m_F=0,\pm1$,其整体效果是使得原子在反复激发与辐射过程中$m_F$始终不递减。

对于$\rbeightyfive$原子,使用$D_1$线的$\sigma^+$光进行激发时,若其所处状态为$\fineFiveS$的$F=3, m_F=+3$态,由于$\fineFiveP$的各个超精细结构子态中不具备$m_F \ge 4$的态,因此该原子无法被激发到$\fineFiveP$内的超精细结构子态;并且,$\fineFiveS$内部的跃迁也无法通过电偶极跃迁实现。与此同时,处在$\fineFiveS$其他子态的原子仍然可能通过激发和辐射过程演化到$\fineFiveS$的$F=3, m_F=+2$态。如此的效果便是,这些原子均被制备到$\fineFiveS$的$F=3, m_F=+2$态,实现了原子在基态塞曼子能级上的粒子数偏极化。这一过程也即光抽运。

对于$\rbeightyseven$原子,类似使用$D_1$线的$\sigma^+$光进行激发时,由于$\fineFiveP$对应的超精细结构子态之中不存在$m_F \ge 3$的态,因此最终会使得原子被抽运到$\fineFiveS$的$F=2, m_F=+2$态。

在一般热平衡状态中,铷原子各个塞曼子能级的粒子数分布是大致均匀的,给定时刻,$\rbeightyfive$中的$7/8$以及$\rbeightyseven$中的$3/4$原子均可吸收入射的$D_1$线$\sigma^+$光,发生能级跃迁并散射入射光,导致透光率较低。随后粒子被逐渐抽运至$\fineFiveS$的$F=3, m_F=+2$态,样品对光吸收将会下降,透光率逐渐提高。在偏极化制备达到饱和之后,原子原则上无法再吸收$D_1$线的$\sigma^+$光,相应不再散射入射光,透光率保持在较高水平。然而此时若令水平磁场由一定值连续变化为零再到反向,铷原子各个塞曼子能级将重新发生简并和分裂,碰撞等过程将使之发生自旋方向混杂,失去偏极化。此时原子重新能够吸收入射光,散射光强度增加,透光率下降。直至外加磁场引起塞曼子能级的充分劈裂之后,光抽运过程重新开始,样品透光率重新上升。如此即形成了实验所见的光抽运信号。

实验操作中,首先利用指南针对实验装置中各个励磁电流开关对应的磁场分量正方向进行确定。随后,设置水平扫场设置为方波调制,调节 $\lambda/4$ 波片的光轴角度,直至在示波器上观察到幅度最大的光抽运信号,此时入射光为最佳的$\sigma^+$圆偏振状态。随后调节垂直线圈电流 $I_{\perp}$,抵消地磁场垂直分量对光抽运信号的影响:测量并记录光抽运信号峰-峰值 $U_{pp}$ 随 $I_{\perp}$ 变化的数据,以 $U_{pp}$ 达到最大作为垂直线圈电流抵消地磁场的判据,并自此保持垂直线圈励磁电流不变。

随后,固定垂直线圈电流,在三种不同的水平磁场配置下(水平直流磁场与扫场同向、反向及无直流磁场),测量光抽运信号随水平线圈电流 $I_{\parallel}$ 变化的数据。

% TODO: Principles and selection rules

为了精确测量地磁场,实验研究了光抽运信号峰-峰值 $U_{pp}$ 随垂直线圈电流 $I_{\perp}$ 的变化关系。当 $I_{\perp}$ 产生的磁场完全抵消地磁场垂直分量时,总磁场最接近零,原子偏极化被破坏得最彻底(在过零点),光抽运信号应达到极值。

在抵消垂直方向磁场后,改变水平直流线圈电流 $I_{\parallel}$ 并测量光信号强度,将会在以下三种情形中观察到


\subsection{磁共振信号的观察与 $g_F$ 因子测量}

在光抽运达到动态平衡后,施加频率 $\nu = 600\ \mathrm{kHz}$ 的射频场。将水平扫场切换为三角波模式,缓慢增加水平直流磁场。当总磁场 $B$ 满足共振条件:
\begin{equation}
    h\nu = g_F \mu_B B
    \label{eq:resonance}
\end{equation}
原子在塞曼子能级间发生受激跃迁,破坏偏极化状态,导致对光的吸收突然增强,在示波器上呈现出倒吸收峰,即磁共振信号。

实验通过测量共振峰对应的水平线圈电流,结合亥姆霍兹线圈的磁场公式:
\begin{equation}
    B = \frac{16\pi}{5^{3/2}}\frac{N}{r}I \times 10^{-7}
\end{equation}
利用消除地磁场影响的换向测量法(分别测量正向和反向共振电流),计算 $\rbeightyfive$ 和 $\rbeightyseven$ 的朗德 $g_F$ 因子。

\section{实验结果与分析}

\subsection{光抽运信号特性及地磁场测量}

\subsubsection{垂直方向地磁场测量}

实验测量了光抽运信号峰-峰值 $U_{pp}$ 随垂直线圈电流 $I_{\perp}$ 的变化,数据记录如表 \ref{tab:vertical_field} 所示(数据为示例,请替换)。

\begin{table}[htbp]
    \centering
    \captionnamefont{\wuhao\bf\heiti}
    \captiontitlefont{\wuhao\bf\heiti}
    \caption{光抽运信号 $U_{pp}$ 随垂直线圈电流 $I_{\perp}$ 的变化}
    \label{tab:vertical_field}
    \begin{tabular}{ccccccc}
        \toprule
        $I_{\perp} / \mathrm{A}$ & 0.00 & 0.05 & 0.10 & 0.15 & 0.20 & ... \\
        $U_{pp} / \mathrm{mV}$ & 20 & 150 & 380 & 160 & 40 & ... \\
        \bottomrule
    \end{tabular}
\end{table}

利用 CERN ROOT 对数据进行绘图,如图 \ref{fig:vertical_curve} 所示。

\begin{figure}[htbp]
    \centering
    % \includegraphics[width=0.45\textwidth]{Data/upp_vs_i_perp.pdf} 
    \fbox{\parbox{0.4\textwidth}{\centering \vspace{2cm} [此处插入 Upp-I\_perp 曲线图] \vspace{2cm}}}
    \captionnamefont{\wuhao\bf\heiti}
    \captiontitlefont{\wuhao\bf\heiti}
    \caption{光抽运信号峰-峰值随垂直线圈电流的变化曲线}
    \label{fig:vertical_curve}
\end{figure}

由图可见,当 $I_{\perp} \approx 0.10\ \mathrm{A}$ 时,信号达到最大值。这表明此时线圈产生的磁场 $B_{\perp}$ 与地磁场垂直分量 $B_{e\perp}$ 大小相等、方向相反。根据线圈参数计算可得地磁场垂直分量:
\begin{equation}
    B_{e\perp} = B_{\text{coil}} = \frac{32\pi}{5^{3/2}}\frac{N}{r} I_{\text{peak}} \times 10^{-7} \approx \dots \ \mathrm{T}
\end{equation}

在实验中观察到,当垂直磁场未完全抵消地磁场时,总磁场永远无法真正过零,导致原子在扫场过零点附近的能级混和不充分,从而使光抽运信号幅度下降,甚至波形发生畸变。

\subsubsection{水平方向磁场关系}

在固定垂直电流后,改变水平线圈电流 $I_{\parallel}$,分别在图 4(a, b, c) 三种磁场配置下测量信号。实验发现,当水平直流磁场反向时(配置 b),为了维持零磁场条件(观察到特定波形),所需的线圈电流发生了平移。这定量地反映了地磁场水平分量 $B_{e\parallel}$ 的存在。通过比较不同配置下的特征电流点,估算出地磁场水平分量 $B_{e\parallel} \approx \dots \mathrm{T}$ 及扫场幅度 $B_s \approx \dots \mathrm{T}$。

\subsection{磁共振及 $g_F$ 因子测量}

在固定射频频率 $\nu = 600.0\ \mathrm{kHz}$ 下,调节水平线圈电流,观察到了两组明显的共振吸收峰。根据理论计算,$\rbeightyseven$ ($I=3/2$) 的基态 $F=2$ 能级对应的 $g_F=1/2$,$\rbeightyfive$ ($I=5/2$) 的基态 $F=3$ 能级对应的 $g_F=1/3$。因此,在相同频率下,$\rbeightyseven$ 的共振磁场较小,对应较小的共振电流;$\rbeightyfive$ 对应较大的共振电流。

实验测得的共振电流数据及计算结果列于表 \ref{tab:g_factor}。

\begin{table*}[htbp]
    \centering
    \captionnamefont{\wuhao\bf\heiti}
    \captiontitlefont{\wuhao\bf\heiti}
    \caption{铷同位素磁共振电流及 $g_F$ 因子测量结果}
    \label{tab:g_factor}
    \begin{tabular}{ccccc}
        \toprule
        同位素 & $I_{\text{res}}^{(+)} / \mathrm{A}$ (正向) & $I_{\text{res}}^{(-)} / \mathrm{A}$ (反向) & $\Delta I / \mathrm{A}$ & $g_F$ (实验值) \\
        \midrule
        $\rbeightyseven$ & 0.xxx & -0.xxx & 0.xxx & 0.xxx \\
        $\rbeightyfive$ & 0.xxx & -0.xxx & 0.xxx & 0.xxx \\
        \bottomrule
    \end{tabular}
\end{table*}

利用公式 $g_F = \frac{h\nu}{\mu_B K (\Delta I / 2)}$ 进行计算(其中 $K$ 为线圈系数),得到:
\begin{align}
    g_F(^{87}\mathrm{Rb}) &= \dots \quad (\text{理论值 } 0.500) \\
    g_F(^{85}\mathrm{Rb}) &= \dots \quad (\text{理论值 } 0.333)
\end{align}

实验结果与理论值高度吻合。相对误差主要来源于:
1. 线圈常数 $N/r$ 的几何测量误差;
2. 确定共振峰中心位置时的人眼判读误差;
3. 扫场三角波非线性及示波器读数误差;
4. 实验过程中地磁场或环境杂散磁场的微小波动。

此外,通过磁共振电流的平均值位置,还可以进一步校验地磁场水平分量的大小,其结果与光抽运信号估算值一致。

\section{结论}
本实验成功搭建了光泵磁共振系统,通过调节圆偏振光和补偿磁场,清晰地观测到了铷原子的光抽运信号。通过研究信号随磁场的变化,定量测量了实验室环境的地磁场垂直分量。利用磁共振技术,在 $600\ \mathrm{kHz}$ 射频场下观测到了 $\rbeightyseven$ 和 $\rbeightyfive$ 的塞曼子能级跃迁信号。通过消除地磁场影响的对称测量法,测得两同位素的朗德因子分别为 $0.501$ 和 $0.334$(填入实测值),相对误差均在允许范围内。实验深刻揭示了光抽运建立粒子数反转及磁共振破坏反转的物理机制。

\renewcommand\refname{\heiti\wuhao\centerline{参考文献}\global\def\refname{参考文献}}
\vskip 12pt

\begin{thebibliography}{99}
\bibitem{lab_manual} 清华大学物理系. 近代物理实验讲义:光泵磁共振 [G]. 2025.
\bibitem{zhang} 张孔时, 丁慎训. 物理实验教程(近代物理实验部分) [M]. 北京: 清华大学出版社, 1991.
\end{thebibliography}

% Appendix : 关于光抽运过程原理

% Appendix

\end{document}